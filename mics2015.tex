% This is sigproc-sp.tex -FILE FOR V2.6SP OF ACM_PROC_ARTICLE-SP.CLS
% OCTOBER 2002
%
% It is an example file showing how to use the 'acm_proc_article-sp.cls' V2.6SP
% LaTeX2e document class file for Conference Proceedings submissions.
% ----------------------------------------------------------------------------------------------------------------
% This .tex file (and associated .cls V2.6SP) *DOES NOT* produce:
%       1) The Permission Statement
%       2) The Conference (location) Info information
%       3) The Copyright Line with ACM data
%       4) Page numbering
%
%  However, both the CopyrightYear (default to 2002) and the ACM Copyright Data
% (default to X-XXXXX-XX-X/XX/XX) can still be over-ridden by whatever the author
% inserts into the source .tex file.
% e.g.
% \CopyrightYear{2003} will cause 2003 to appear in the copyright line.
% \crdata{0-12345-67-8/90/12} will cause 0-12345-67-8/90/12 to appear in the copyright line.
%
% ---------------------------------------------------------------------------------------------------------------
% It is an example which *does* use the .bib file (from which the .bbl file
% is produced).
% REMEMBER HOWEVER: After having produced the .bbl file,
% and prior to final submission,
% you need to 'insert'  your .bbl file into your source .tex file so as to provide
% ONE 'self-contained' source file.
%
% Questions regarding SIGS should be sent to
% Adrienne Griscti ---> griscti@acm.org
%
% Questions/suggestions regarding the guidelines, .tex and .cls files, etc. to
% Gerald Murray ---> murray@acm.org 
%
% For tracking purposes - this is V2.6SP - OCTOBER 2002


\documentclass[12pt]{article}

\setlength{\oddsidemargin}{0in}
\setlength{\evensidemargin}{0in}
\setlength{\topmargin}{0in}
\setlength{\headheight}{0in}
\setlength{\headsep}{0in}
\setlength{\textwidth}{6in}
\setlength{\textheight}{9in}
\setlength{\parindent}{0in} 

\usepackage{graphicx} %For jpg figure inclusion
\usepackage{times} %For typeface
\usepackage{epsfig}
\usepackage{color} %For Comments
%\usepackage[all]{xy}
\usepackage{float}
%\usepackage{subfigure} 
\usepackage{hyperref}
\usepackage{url}
\usepackage{parskip}

%% Elena's favorite green (thanks, Fernando!)
\definecolor{ForestGreen}{RGB}{34,139,34}
\definecolor{BlueViolet}{RGB}{138,43,226}
%Uncomment this if you want to show work-in-progress comments
\newcommand{\comment}[1]{{\bf \tt  {#1}}}
% Uncomment this if you don't want to show comments
%\newcommand{\comment}[1]{}
\newcommand{\emcomment}[1]{\textcolor{ForestGreen}{\comment{Elena: {#1}}}}
\newcommand{\todo}[1]{\textcolor{blue}{\comment{To Do: {#1}}}}
\newcommand{\pscomment}[1]{\textcolor{red}{\comment{Paul: {#1}}}}
\newcommand{\mmcomment}[1]{\textcolor{magenta}{\comment{Max: {#1}}}}
\newcommand{\escomment}[1]{\textcolor{BlueViolet}{\comment{Emma: {#1}}}}

%%%%%%%%%%%%%%%%%%%%%%%%%%%%%%%%%%%%%%%%%%

\begin{document}
\pagestyle{plain}
%
% --- Author Metadata here ---
%\conferenceinfo{WOODSTOCK}{'97 El Paso, Texas USA}
%\setpagenumber{50}
%\CopyrightYear{2002} % Allows default copyright year (2002) to be
%over-ridden - IF NEED BE. 
%\crdata{0-12345-67-8/90/01}  % Allows default copyright data
%(X-XXXXX-XX-X/XX/XX) to be over-ridden. 
% --- End of Author Metadata ---

\title{Developing Beginner-Friendly User Interactions for the Clojure Programming Language}
%\subtitle{[Extended Abstract \comment{DO WE NEED THIS?}]
%\titlenote{}}
%
% You need the command \numberofauthors to handle the "boxing"
% and alignment of the authors under the title, and to add
% a section for authors number 4 through n.
%
% Up to the first three authors are aligned under the title;
% use the \alignauthor commands below to handle those names
% and affiliations. Add names, affiliations, addresses for
% additional authors as the argument to \additionalauthors;
% these will be set for you without further effort on your
% part as the last section in the body of your article BEFORE
% References or any Appendices.

\author{
Henry Fellows, Aaron Lemmon, Max Magnuson, \\
	Emma Sax, Paul Schliep, and Elena Machkasova \\
Computer Science Discipline \\
University of Minnesota Morris\\
Morris, MN 56267\\
fello056@umn.edu, lemmo031@umn.edu, magnu401@umn.edu, \\
	saxxx027@umn.edu, schli202@umn.edu, elenam@umn.edu
}
\date{}
\maketitle
\thispagestyle{empty}

\section*{\centering Abstract}
The abstract
\escomment{Is this supposed to be the abstract that we submitted? The
  one that's three paragraphs long?}
\emcomment{No, usually it's different from the abstract submitted for
  review. An abstract is usually written when the paper is nearly
  finished.}

\newpage
\setcounter{page}{1}

\section{Introduction and Background}\label{sec:intro}
%\escomment{All of my titles and stuff are optional. I'm just trying to give us a template.}
\subsection{Overview of Clojure}\label{sec:clojure}
\emcomment{Emma - many thanks for getting things started! Below are
  comments. The majority of them refer to making this material
  understandable to the target audience (i.e not using terms that
  CSci undergrads typically don't know. Otehrwise things are well
  organized and well written.}
Clojure, developed in 2007 by Rich Hickey, is a dynamic, functional
programming language in the Lisp family. 
\emcomment{Use simpler, more concrete terminology: there are many ways
  in which JVM can be utilized. Probably need to explain that the JVM
  is Clojure interpreter, mention and explain REPL somewhere in this paragraph.}
\emcomment{Need to explain what you mean by dynamic}
It utilizes the Java Virtual
Machine, which means that it compiles to JVM bytecode, and it offers
easy 
\emcomment{accessibilty?} 
availability of Java frameworks, 
\emcomment{This wouldn't make any sense to the target audience and is
  not needed (perhaps with the exception of multithreading:}
including optional type hints
and type inference and easy multithreading options. All the while,
Clojure remains completely dynamic.  
\emcomment{This is a strange wording: what does it mena to ``remain''
  in this context? And you already said it was dynamic.}

Because Clojure is a functional language, it puts strong emphasis on
immutable data types. An immutable data type means data types cannot
be changed, unlike \emcomment{in? Also, why ``many'' and not all of
  them?} many imperative programming languages. When using
Clojure, in order to change a data type, an entirely new data item
must be made. Immutable data types are practical for novice
programmers because immutable data types are useful for avoiding side
effects in functions. \emcomment{I would first explain what side
  effects are, and only then why they complicate things for novice
  programmers}
A side effect is when a function alters memory
or interacts outside of its scope instead of returning a value. Side
effects can make debugging or resolving errors or failures more
difficult. This is because when a function interacts with other
functions \emcomment{``other'' usually means ``than itself''} than
itself, when it is not supposed to, any issues in the 
code can be spread out throughout the program. The reduction of side
effects means problems with the code are easier to find and fix. 

Clojure, like other Lisp languages, uses prefix notation. This means
that function calls use parenthesis, followed by the function name,
and then any parameters: 
\begin{verbatim}
	(<function-name> <argument 1> <argument 2>)
\end{verbatim}

Even built-in functions, such as mathematical functions, use this form:
\begin{verbatim}
	(+ 5 5)
	-> 10
\end{verbatim}

Note that \texttt{->} signals the result of the
function. \emcomment{We indicate the result of computations in Clojure
  interpreter as \texttt{->} -- plus we need to explain REPL and
  interpreter.}

\emcomment{I don't think we need syntax for interop. I would comment
 out the example for now.}
Clojure also has easy accessibility to Java functions and Java
interoperability. This means that any Java method can be called just
like normal Clojure functions: 
\begin{verbatim}
	(.methodName object *arguments)
\end{verbatim}

\emcomment{Thsi is true, but probably not useful for our readers}
As well as the use of Java methods, Clojure can also use macros, Java
utilities, concurrency and multithreading, and even certain Java
objects to enhance the abilities of using Clojure. 

\emcomment{before we get into anonymous functions, we need to show how
to define regular functions (general syntax and a couple of examples)}
Clojure supports anonymous functions. Anonymous functions allow
programmers to quickly make rare \emcomment{not sure what you mean by
  ``rare'' here. Those that aren't predefined?}  functions when
needed. 
However, these
functions are not  stored, and so the program would not be able to
called \emcomment{call them?} by name after the usage. 
The following example is an anonymous
function that takes a number and increments it by one: 
\begin{verbatim}
	(fn [number] (+ number 1))
\end{verbatim}

In anonymous functions, anything in the brackets that is following the
\texttt{fn} is arguments. 
\emcomment{and in regular functions as well} 
After the arguments in \texttt{[]}, is the
body. If the program wanted to be able to call this function multiple
times, it would make more sense to simply put a \texttt{def} with a
name in front: 
\emcomment{It's not quite that: firstly, it's defn, and not def, but
  you also need to explain where the function name is}
\begin{verbatim}
	(def increment-number [number] (+ number 1))
\end{verbatim}

Now, the program can call this function any time by using the name \texttt{increment-number}.

Clojure also has a variety of different types of data structures. All
of Clojure's collections are both immutable and persistent. 
\emcomment{I wouldn't go into persistent, just talk about immutable.}
This means
that the data in the structures cannot be modified, and that each
structure preserves the previous version of itself. The different
types of structures Clojure uses are lists (denoted by \texttt{()}),
sets (denoted by \texttt{\#\{\}}), vectors (denoted by \texttt{[]}),
and hashmaps (denoted by \texttt{\{\}}). The first three types of data
structures listed above are normal \emcomment{don't use the word
  ``normal'' in a research paper} data structures: 
\begin{verbatim}
	(1 2 "foo" :a 9 "bar")
\end{verbatim}
\emcomment{I am not sure which of data structures we will actually need; we
  never defined keywords, and I am not sure we will need those (but if
  we do, we will need to explain them). We will definitely need
  vectors and lists and hashmaps. Sets probably not.}

They \emcomment{Not sure what ``they'' referes to; most structures can
hold any amount (any number, to be more precise) of values, nil isn't
explained, and probably needs to 
be, but I am not sure; nil isn't an amount of anything.} can hold any
amount, including \texttt{nil}, of values of any 
type 
\emcomment{we need to mention earlier that all data structures are
  untyped: not unique to hashmaps}
(a data structure is untyped, which means it does not care if
types of values in one structure as mismatched). However, hashmaps are
unique in the fact that a hashmap is a collection of key-value pairs: 
\begin{verbatim}
	{:a 1 :b 2 :c 3}
\end{verbatim}
\emcomment{It's unclear from just the syntax what it means to be a
  collection of key/value pairs.}

In hashmaps, usually keywords, indicated by a colon in front of a
simple indicating word, are used as keys, and the value is what the
key is pointing to. 
\emcomment{keys aren't pointing to anything, they are bound to something}
In the hashmap above, the keywords are
\texttt{:a}, \texttt{:b}, and \texttt{:c}, and they are pointing to
the values \texttt{1}, \texttt{2}, and \texttt{3}. 

Just like variables and values, data structures can also be bound to a
variable name: 
\begin{verbatim}
	(def myhashmap {:a 1 :b 2 :c 3})
\end{verbatim}

This enables the data structures to be called, traversed, etc by the
use of a single name. \emcomment{I am not sure this addds anything; an
example would be better.}

\subsection{Overview of ClojurEd}\label{sec:project}

\section{Error Messages}\label{sec:errors}

\section{Technical Setup}\label{sec:technical}

\section{Conclusions}\label{sec:conclusion}

\section{References?}\label{sec:reference}
\escomment{unsure if we actually put references directly into the
  paper or not...}
\emcomment{Emma - Don't worry about it for now. Do you have a reference you
  would like to add?}

% That's all folks!
\end{document}